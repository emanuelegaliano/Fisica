\chapter{Dinamica del punto materiale}

La dinamica del punto materiale studia il moto dei corpi materiali sotto l'azione di forze esterne. A differenza della cinematica, che descrive il movimento prescindendo dalle cause che lo generano, la dinamica si propone di individuare e analizzare le \emph{interazioni fisiche} responsabili delle variazioni dello stato di moto dei corpi.

Nello studio della dinamica, i corpi reali vengono spesso schematizzati come \textbf{punti materiali}, ossia oggetti dotati di massa ma privi di dimensioni spaziali apprezzabili rispetto al fenomeno considerato. Tale modello consente di semplificare l’analisi del moto, concentrandosi esclusivamente sugli effetti delle forze applicate al corpo.

L’obiettivo fondamentale della dinamica è stabilire una relazione quantitativa tra le \textbf{forze agenti} su un corpo e il suo \textbf{moto}, in particolare attraverso lo studio delle variazioni della velocità nel tempo. Questo legame è formalizzato dai \emph{principi della dinamica}, enunciati da Newton, che costituiscono il fondamento della meccanica classica.

\section{Il concetto di forza}

In dinamica, il concetto centrale è quello di \textbf{forza}. In modo intuitivo, una forza rappresenta un’interazione tra corpi capace di modificare lo stato di moto di un corpo oppure di deformarlo. Dal punto di vista fisico, una forza è dunque la causa delle variazioni del moto osservate sperimentalmente.

La forza è una \textbf{grandezza vettoriale}, caratterizzata da:
\begin{itemize}
    \item un \textbf{modulo}, che ne misura l’intensità;
    \item una \textbf{direzione};
    \item un \textbf{verso};
    \item un \textbf{punto di applicazione}.
\end{itemize}
Per descrivere correttamente l’azione di una forza su un corpo è necessario specificare tutte queste caratteristiche.

Nel Sistema Internazionale, l’unità di misura della forza è il \textbf{newton} (N), definito come la forza che, applicata a un corpo di massa pari a $1\,\text{kg}$, gli imprime un’accelerazione di $1\,\text{m/s}^2$.

Quando su un corpo agiscono più forze contemporaneamente, l’effetto complessivo sul moto è determinato dalla \textbf{forza risultante}, ottenuta come somma vettoriale di tutte le forze applicate:
\begin{equation}
\boxed{
\vec{F}_{\text{tot}} = \sum_i \vec{F}_i
}
\end{equation}
\addcontentsline{myequations}{equ}{Forza totale agente su un corpo}

È la forza totale agente sul corpo a determinare le eventuali variazioni del suo stato di moto.

\section{I principi della dinamica}

I principi della dinamica, formulati da Isaac Newton, costituiscono il fondamento della meccanica classica. Essi stabiliscono le leggi che governano il moto dei corpi in relazione alle forze che agiscono su di essi e risultano validi, con ottima approssimazione, per sistemi macroscopici che si muovono a velocità molto inferiori a quella della luce.

\subsection{Primo principio della dinamica}

Il primo principio della dinamica, noto anche come \textbf{principio di inerzia}, afferma che:

\begin{quote}
Un corpo permane nel suo stato di quiete o di moto rettilineo uniforme finché una forza esterna risultante non interviene a modificarne lo stato.
\end{quote}

Questo principio introduce il concetto di \textbf{inerzia}, ossia la tendenza dei corpi a opporsi alle variazioni del proprio stato di moto. In assenza di forze esterne, oppure quando la forza risultante agente su un corpo è nulla, il corpo non subisce alcuna accelerazione.

\textit{Il primo principio permette di identificare i sistemi di riferimento \textbf{inerziali}: sono tali quei sistemi nei quali un corpo non soggetto a forze si muove di moto rettilineo uniforme.}

\subsection{Secondo principio della dinamica}

Il secondo principio della dinamica stabilisce una relazione quantitativa tra la forza risultante applicata a un corpo e l’accelerazione che esso acquista. Esso afferma che:

\begin{quote}
L’accelerazione di un corpo è direttamente proporzionale alla forza risultante che agisce su di esso ed è inversamente proporzionale alla sua massa.
\end{quote}

In forma matematica, il secondo principio si esprime come:
\begin{equation}
\boxed{
\vec{F}_{\text{tot}} = m \vec{a}
}
\end{equation}
\addcontentsline{myequations}{equ}{Secondo principio della dinamica}

dove $\vec{F}_{\text{tot}}$ è la forza risultante agente sul corpo, $m$ è la massa del corpo e $\vec{a}$ è l’accelerazione prodotta.

La massa rappresenta una misura dell’inerzia del corpo: a parità di forza applicata, un corpo di massa maggiore subisce un’accelerazione minore. In generale, il problema fondamentale della dinamica consiste nel determinare il \textbf{moto di un corpo}, ossia la sua legge oraria $\vec{x}(t)$, a partire dalla conoscenza delle forze agenti su di esso. Poiché l’accelerazione è la derivata seconda della posizione rispetto al tempo, il secondo principio della dinamica conduce, in generale, a un’equazione differenziale del secondo ordine. La determinazione del moto richiede quindi la risoluzione di tale equazione, una volta assegnate le condizioni iniziali.

\textit{Il secondo principio della dinamica mette in relazione le grandezze fondamentali della meccanica classica: forza, massa e accelerazione. Esso costituisce la base per l’analisi quantitativa del moto dei corpi sotto l’azione di forze esterne.}

\subsection{Terzo principio della dinamica}

Il terzo principio della dinamica, detto \textbf{principio di azione e reazione}, afferma che:

\begin{quote}
Se un corpo A esercita una forza su un corpo B, allora il corpo B esercita simultaneamente su A una forza uguale in modulo e direzione, ma opposta in verso.
\end{quote}

Le due forze di azione e reazione costituiscono una coppia e agiscono sempre su \emph{corpi diversi}. Per questo motivo, esse non si annullano a vicenda e non violano il secondo principio della dinamica.

\textit{Il terzo principio evidenzia che le forze sono sempre il risultato di un’interazione reciproca tra corpi e che non esistono forze isolate.}

\section{Forze nella meccanica classica}

In generale, la forza agente su un punto materiale può dipendere dalla posizione del corpo nello spazio. In tal caso, il secondo principio della dinamica assume la forma di un’equazione differenziale in cui la forza non è costante, rendendo la determinazione analitica della legge oraria più complessa. Nei casi più semplici, come quello della forza peso, la forza può essere invece considerata costante, permettendo una integrazione diretta delle equazioni del moto.

Nello studio della dinamica del punto materiale, alcune forze compaiono in modo ricorrente e permettono di costruire modelli semplici ma molto efficaci. In questa sezione introduciamo le forze fondamentali e, quando possibile, ricaviamo le corrispondenti \textbf{leggi orarie} a partire dal secondo principio della dinamica.

\subsection{Forza peso}

\begin{figure}[htbp]
    \centering
    \includegraphics[width=0.4\textwidth]{images/forza_peso.png}
    \caption{Rappresentazione della forza peso $\vec{P}$ agente su un punto materiale di massa $m$.} 
    \label{fig:forza_peso}
\end{figure}

La \textbf{forza peso} è la forza gravitazionale esercitata dalla Terra su un corpo di massa $m$. In prossimità della superficie terrestre può essere considerata costante in modulo e direzione:
\begin{equation}
\boxed{
\vec{P}=m\vec{g},
}
\end{equation}
\addcontentsline{equ}{myequations}{Forza peso}
dove $\vec{g}$ è l'accelerazione di gravità che vale $\approx 9.81 m/s^2$ (diretta verticalmente verso il basso).

\paragraph{Accelerazione di gravità indipendente dalla massa.} 
Applicando il secondo principio della dinamica:
\[
m\vec{a}=\vec{P}=m\vec{g}
\qquad \Longrightarrow \qquad
\vec{a}=\vec{g}.
\]
Si ottiene quindi un risultato fondamentale: L'accelerazione di un corpo soggetto alla sola forza peso è \textbf{costante} ed è \textbf{indipendente dalla massa} del corpo.

\subsubsection*{Derivazione delle leggi orarie del moto}

Consideriamo un punto materiale soggetto a una forza costante diretta lungo l’asse $x$:
\[
\vec{F} = (F,0,0), \qquad \vec{F}=\text{costante}.
\]

Applicando il secondo principio della dinamica in componenti si ha:
\[
F_x = F = m a_x = m \frac{d^2 x(t)}{dt^2},
\qquad
F_y = 0 = m a_y,
\qquad
F_z = 0 = m a_z.
\]

Da cui segue immediatamente che:
\[
a_y = 0, \qquad a_z = 0,
\]
e quindi le componenti del moto lungo $y$ e $z$ sono di tipo rettilineo uniforme:
\[
v_y = \text{costante}, \qquad v_z = \text{costante}.
\]

\paragraph{Componente $x$.}

Dalla seconda legge di Newton lungo $x$, $F_x = m a_x$, si ottiene:
\[
\frac{F}{m} = a_x = \frac{d^2 x(t)}{dt^2} = \frac{d v_x(t)}{dt}.
\]

Integrando nel tempo:
\[
\int_0^t \frac{F}{m}\,dt = \int_{v_{x0}}^{v_x(t)} dv_x
\quad\Longrightarrow\quad
\frac{F}{m}t = v_x(t) - v_{x0}.
\]

Si ottiene quindi:
\[
v_x(t) = v_{x0} + \frac{F}{m}t.
\]

Poiché:
\[
v_x(t) = \frac{dx(t)}{dt},
\]
integrando nuovamente:
\[
\int_{x_0}^{x(t)} dx
= \int_0^t v_{x0}\,dt + \int_0^t \frac{F}{m}t\,dt.
\]

Da cui:
\[
x(t) - x_0 = v_{x0}t + \frac{1}{2}\frac{F}{m}t^2,
\]
ossia:
\[
x(t) = x_0 + v_{x0}t + \frac{1}{2}\frac{F}{m}t^2.
\]

\medskip
Si conclude che, nel caso di una forza costante applicata lungo un asse, il moto lungo tale direzione è \textbf{uniformemente accelerato}, mentre lungo le direzioni perpendicolari è \textbf{rettilineo uniforme}.

\subsection{Forza elastica (legge di Hooke)}

\begin{figure}[htbp]
    \centering
    \includegraphics[width=0.8\textwidth]{images/forza_elastica.png}
    \caption{Sistema massa--molla su piano orizzontale: una massa $m$ è collegata a una molla ideale e può muoversi lungo l’asse $x$. La posizione di equilibrio è indicata con $x_0$; per uno spostamento $\Delta x = x - x_0$ la molla esercita una forza elastica diretta verso la posizione di equilibrio.}
    \label{fig:forza_elastica}
\end{figure}

La \textbf{forza elastica} è una \textbf{forza di richiamo}: tende a riportare il punto materiale verso la \textbf{posizione di equilibrio}. Essa è la forza esercitata dalla molla quando questa viene deformata (allungata o compressa).

Nel caso unidimensionale (moto lungo l’asse $x$), indicando con $x_0$ la posizione di equilibrio e con $\Delta x = x - x_0$ lo spostamento dall’equilibrio, la legge di Hooke afferma che:
\begin{equation}
\boxed{
F_e = -k\,\Delta x = -k(x-x_0),
}
\end{equation}
\addcontentsline{equ}{myequations}{Forza elastica}

dove $k$ è la \textbf{costante elastica} della molla. Il segno meno indica che la forza è sempre opposta allo spostamento: se $\Delta x>0$ la forza è diretta verso sinistra, se $\Delta x<0$ è diretta verso destra.

\paragraph{Derivazione della legge oraria (moto armonico).}

Applichiamo il secondo principio della dinamica lungo l’asse $x$:
\[
\sum F_x = m\,\frac{d^2 x(t)}{dt^2}.
\]
Se l’unica forza lungo $x$ è la forza elastica, allora:
\[
-k(x(t)-x_0)=m\,\frac{d^2 x(t)}{dt^2}.
\]
Portando tutto a primo membro:
\[
\frac{d^2 x(t)}{dt^2}+\frac{k}{m}\bigl(x(t)-x_0\bigr)=0.
\]

È spesso comodo riscrivere l’equazione in termini dello spostamento dall’equilibrio $\Delta x(t)=x(t)-x_0$:
\[
\frac{d^2 \Delta x(t)}{dt^2}+\omega^2\,\Delta x(t)=0,
\qquad \text{con } \omega=\sqrt{\frac{k}{m}}.
\]
Questa è l’equazione del moto \textbf{armonico} (moto periodico).

Una soluzione generale può essere scritta come:
\[
\Delta x(t)=A\cos(\omega t+\varphi),
\]
dove $A$ è l’ampiezza dell’oscillazione e $\varphi$ è la fase iniziale. Di conseguenza:
\[
x(t)=x_0 + A\cos(\omega t+\varphi).
\]

Il moto è periodico con periodo:
\[
T=\frac{2\pi}{\omega}=2\pi\sqrt{\frac{m}{k}}.
\]

Caso particolare: se il corpo viene lasciato da fermo con elongazione iniziale $L$ rispetto all’equilibrio, cioè $\Delta x(0)=L$ e $v(0)=0$, allora $\varphi=0$ e:
\[
\Delta x(t)=L\cos(\omega t),
\qquad
x(t)=x_0+L\cos(\omega t).
\]

\begin{figure}[htbp]
    \centering
    \includegraphics[width=0.7\textwidth]{images/grafico_oscillazione_armonica.png}
    \caption{Andamento temporale dello spostamento $\Delta x(t)$ nel moto armonico: l’oscillazione è periodica attorno alla posizione di equilibrio, con ampiezza $L$ e periodo $T=2\pi\sqrt{m/k}$.}
    \label{fig:grafico_oscillazione_armonica}
\end{figure}

\subsection{Forza viscosa}

La \textbf{forza viscosa} è una forza dissipativa che si manifesta quando un corpo si muove all’interno di un fluido (aria, acqua, ecc.). Essa è diretta in verso opposto alla velocità del corpo e, nel regime di basse velocità, è proporzionale al modulo della velocità stessa:
\begin{equation}
\boxed{
    \vec{F}_v = -\beta\,\vec{v},
}
\end{equation}
\addcontentsline{equ}{myequations}{Forza viscosa}
dove $\beta$ è una costante positiva che dipende dalle proprietà del fluido e dalle dimensioni del corpo.

\paragraph{Velocità limite.}

Consideriamo un corpo che si muove verticalmente sotto l’azione della forza peso e della forza viscosa. Scegliamo l’asse $y$ orientato verso il basso. Le forze agenti lungo $y$ sono:
\[
\vec{P} = m\vec{g}, 
\qquad
\vec{F}_v = -\beta \vec{v}.
\]

Dopo un certo intervallo di tempo, il corpo raggiunge una \textbf{velocità limite} $v_L$, che rimane costante. In tale condizione l’accelerazione è nulla e la forza risultante si annulla:
\[
\vec{a}=0
\quad\Longrightarrow\quad
\vec{P} + \vec{F}_v = 0.
\]
Proiettando lungo l’asse $y$:
\begin{equation}
mg - \beta v_L = 0
\quad\Longrightarrow\quad
\boxed{
v_L = \frac{mg}{\beta}.
}
\end{equation}
\addcontentsline{equ}{myequations}{Velocità limite sotto forza viscosa}

La velocità limite non dipende dalla quota iniziale del corpo, ma solo dai parametri fisici del sistema.

\paragraph{Equazione del moto e legge della velocità.}

Prima di raggiungere la velocità limite, il corpo è accelerato. Applicando il secondo principio della dinamica lungo l’asse $y$ si ottiene:
\[
m\frac{dv_y(t)}{dt} = mg - \beta v_y(t).
\]
Questa è un’equazione differenziale del primo ordine. La sua soluzione, imponendo la condizione iniziale $v_y(0)=0$, è:
\[
v_y(t) = v_L\left(1 - e^{-\alpha t}\right),
\qquad
\text{con } \alpha = \frac{\beta}{m}.
\]

Il parametro $\alpha$ indica la rapidità con cui il corpo raggiunge la velocità limite:
\[
\lim_{t\to\infty} v_y(t) = v_L,
\qquad
v_y(0)=0.
\]

La presenza della forza viscosa modifica profondamente il moto rispetto al caso della sola forza peso: l’accelerazione non è costante e il moto non è uniformemente accelerato. La forza viscosa introduce inoltre una dissipazione di energia meccanica.


\subsection{Forza normale}

La \textbf{forza normale} $\vec{N}$ è una forza vincolare esercitata da una superficie su un corpo a contatto con essa. Essa è diretta perpendicolarmente alla superficie di contatto e impedisce al corpo di attraversarla.

In molte situazioni di interesse, come nel caso di un corpo appoggiato su un piano orizzontale, l'accelerazione lungo la direzione verticale è nulla. Applicando il secondo principio della dinamica lungo l'asse verticale si ha:
\begin{equation}
\sum F_y = 0
\quad\Longrightarrow\quad
N - mg = 0
\quad\Longrightarrow\quad
\boxed{
N = mg.
}
\end{equation}
\addcontentsline{equ}{myequations}{Forza normale su un piano orizzontale}

\subsection{Forza di attrito}

La \textbf{forza di attrito} si oppone al moto relativo, o alla tendenza al moto, tra due superfici a contatto. Essa agisce lungo la superficie di contatto ed è diretta in verso opposto alla velocità relativa o alla forza che tende a mettere il corpo in movimento.

\subsubsection{Attrito statico}

Quando il corpo è fermo rispetto alla superficie di contatto, l'attrito è di tipo statico. Il modulo della forza di attrito statico si adatta al valore necessario a mantenere il corpo in quiete, fino a un valore massimo:
\begin{equation}
\boxed{
|\vec{F}_s| \le \mu_s N,
}
\end{equation}
\addcontentsline{equ}{myequations}{Forza di attrito statico}
dove $\mu_s$ è il coefficiente di attrito statico e $N$ è il modulo della forza normale.

\subsubsection{Attrito dinamico}

Quando il corpo è in movimento rispetto alla superficie, l'attrito è di tipo dinamico. In questo caso il modulo della forza di attrito è costante e vale:
\begin{equation}
\boxed{
|\vec{F}_d| = \mu_d N,
}
\end{equation}
\addcontentsline{equ}{myequations}{Forza di attrito dinamico}
dove $\mu_d$ è il coefficiente di attrito dinamico, in genere minore del coefficiente di attrito statico ($\mu_d < \mu_s$). La forza di attrito dinamico è sempre diretta in verso opposto alla velocità del corpo.

\subsubsection*{Esempio: piano inclinato}

\begin{figure}[h!]
    \centering
    \includegraphics[width=0.8\textwidth]{images/piano_inclinato.png}
    \caption{Punto materiale di massa $m$ su un piano inclinato di angolo $\alpha$. Sono rappresentate la forza peso, la reazione normale e la componente tangenziale della forza peso lungo il piano.}
    \label{fig:piano_inclinato}
\end{figure}

Consideriamo un punto materiale di massa $m$ appoggiato su un piano inclinato di angolo $\alpha$ rispetto all’orizzontale. Sul corpo agiscono la forza peso $\vec{P}$, la forza normale $\vec{N}$ e la forza di attrito.

Scomponiamo la forza peso nelle componenti perpendicolare e parallela al piano. Lungo la direzione perpendicolare al piano il corpo non accelera, quindi vale la condizione di equilibrio:
\[
N - mg\cos\alpha = 0
\quad\Longrightarrow\quad
N = mg\cos\alpha.
\]

La componente della forza peso parallela al piano vale invece:
\[
P_{\parallel} = mg\sin\alpha,
\]
ed è la forza responsabile del moto lungo il piano.

\medskip
\noindent
\textbf{Condizione di distacco.}
La forza di attrito statico può assumere valori fino a un massimo:
\begin{equation}
\boxed{
F_s^{\max} = \mu_s N = \mu_s mg\cos\alpha.
}
\end{equation}
\addcontentsline{equ}{myequations}{Condizione di distacco}
Il corpo rimane in quiete se la componente tangenziale della forza peso non supera tale valore, cioè se:
\[
mg\sin\alpha \le \mu_s mg\cos\alpha
\quad\Longrightarrow\quad
\tan\alpha \le \mu_s.
\]
L’angolo $\alpha_c$ tale che $\tan\alpha_c = \mu_s$ prende il nome di \textbf{angolo critico di distacco}.  
Per $\alpha > \alpha_c$ il corpo inizia a muoversi lungo il piano.

\medskip
\noindent
\textbf{Calcolo dell’accelerazione (attrito dinamico).}
Supponiamo ora che il corpo sia in moto lungo il piano e che agisca l’attrito dinamico. Il modulo della forza di attrito dinamico è:
\[
F_d = \mu_d N = \mu_d mg\cos\alpha,
\]
diretta in verso opposto al moto.

Applicando il secondo principio della dinamica lungo la direzione del piano inclinato si ottiene:
\[
mg\sin\alpha - \mu_d mg\cos\alpha = m a.
\]
Da cui segue l’accelerazione del corpo:
\[
a = g\sin\alpha - \mu_d g\cos\alpha
= g\sin\alpha\left(1 - \frac{\mu_d}{\tan\alpha}\right).
\]

\medskip
\noindent
\textbf{Legge oraria del moto.}
Poiché l’accelerazione è costante, il moto lungo il piano è uniformemente accelerato. Supponendo che il corpo parta da fermo, la legge oraria lungo la direzione del piano è:
\begin{equation}
\boxed{
x(t) = x_0 - \frac{1}{2} a t^2
= x_0 - \frac{1}{2} g\sin\alpha
\left(1 - \frac{\mu_d}{\tan\alpha}\right)t^2.
}
\end{equation}
\addcontentsline{equ}{myequations}{Legge oraria del moto su piano inclinato}


Questo esempio mostra come, una volta individuate correttamente le forze agenti sul corpo e le loro componenti lungo la direzione del moto, il secondo principio della dinamica permetta di determinare completamente l’evoluzione temporale del sistema.

\section{Pendolo semplice}

\begin{figure}[htbp]
    \centering
    \includegraphics[width=0.4\textwidth]{images/pendolo_schema.png}
    \caption{Pendolo semplice: un punto materiale di massa $m$ è vincolato a muoversi lungo una circonferenza di raggio $\ell$, collegato a un filo inestensibile. Sono indicati l’angolo $\theta$ rispetto alla verticale, la tensione del filo e la forza peso.}
    \label{fig:pendolo_schema}
\end{figure}

Il \textbf{pendolo semplice} è un sistema costituito da un punto materiale di massa $m$ collegato a un filo ideale (inesistente massa e inestensibile) di lunghezza $\ell$, fissato a un estremo. Il moto del punto materiale avviene su un arco di circonferenza in un piano verticale.

Sul corpo agiscono due forze:
\begin{itemize}
    \item la forza peso $\vec{P}$;
    \item la tensione del filo $\vec{T}$, che rappresenta una forza vincolare.
\end{itemize}

Indichiamo con $\theta$ l’angolo che il filo forma con la verticale. Per convenzione, $\theta>0$ se il corpo si trova a destra della verticale e $\theta<0$ se si trova a sinistra.

\subsection{Forze agenti e direzione del moto}

La tensione del filo è sempre diretta lungo il filo e impedisce al corpo di allontanarsi dal centro della traiettoria. Di conseguenza, essa non contribuisce al moto lungo la direzione tangenziale.

Il moto del pendolo è quindi dovuto esclusivamente alla \textbf{componente tangenziale della forza peso}. Scomponendo $\vec{P}$ lungo le direzioni radiale e tangenziale si ottiene che:
\[
P_{\text{t}} = -mg\sin\theta,
\]
dove il segno meno indica che la forza è diretta in verso opposto all’aumento di $\theta$.

\subsection{Equazione del moto}

Poiché il corpo si muove lungo una circonferenza di raggio $\ell$, la coordinata naturale del moto è l’arco $s=\ell\theta$. L’accelerazione tangenziale vale:
\[
a_{\text{t}} = \frac{d^2 s}{dt^2} = \ell \frac{d^2 \theta(t)}{dt^2}.
\]

Applicando il secondo principio della dinamica lungo la direzione tangenziale:
\[
-mg\sin\theta = m a_{\text{t}} = m \ell \frac{d^2 \theta(t)}{dt^2}.
\]
Dividendo per $m$ e riordinando si ottiene l’equazione del moto del pendolo:
\begin{equation}
\boxed{
\frac{d^2 \theta(t)}{dt^2} + \frac{g}{\ell}\sin\theta(t) = 0.
}
\end{equation}
\addcontentsline{equ}{myequations}{Equazione del moto del pendolo}

Questa equazione differenziale è non lineare e, in generale, non ammette una soluzione analitica semplice.

\subsection{Approssimazione per piccole oscillazioni}

Se l’angolo $\theta$ è sufficientemente piccolo (tipicamente $|\theta|\lesssim 10^\circ$), è possibile utilizzare l’approssimazione:
\[
\sin\theta \simeq \theta.
\]
In questo caso l’equazione del moto diventa:
\begin{equation}
\boxed{
\frac{d^2 \theta(t)}{dt^2} + \frac{g}{\ell}\,\theta(t) = 0,
}
\end{equation}
\addcontentsline{equ}{myequations}{Equazione del moto del pendolo per piccole oscillazioni}
che è l’equazione del \textbf{moto armonico semplice}.

\paragraph{Nota sull'approssimazione.} 
\textit{L'approssimazione nasce dal fatto che la serie di Taylor di $\sin\theta$ attorno a $\theta=0$ può essere approssimata come $\sin\theta \approx \theta$ per angoli piccoli, poiché i termini di ordine superiore diventano trascurabili. Per angoli maggiori, l'errore introdotto dall'approssimazione aumenta, rendendo la soluzione meno accurata.}

\subsection{Soluzione dell’equazione del moto}

Poiché il moto è oscillatorio, cerchiamo una soluzione del tipo:
\[
\theta(t) = A \cos(\omega t + \varphi).
\]
Derivando due volte rispetto al tempo:
\[
\frac{d^2 \theta}{dt^2} = -\omega^2 A \cos(\omega t + \varphi).
\]
Sostituendo nell’equazione del moto si ottiene:
\[
-\omega^2 A \cos(\omega t + \varphi) + \frac{g}{\ell} A \cos(\omega t + \varphi) = 0,
\]
da cui segue la condizione:
\begin{equation}
\boxed{
\omega = \sqrt{\frac{g}{\ell}}.
}
\end{equation}
\addcontentsline{equ}{myequations}{Pulsazione del pendolo semplice}

La pulsazione $\omega$ determina il periodo del moto:
\[
T = \frac{2\pi}{\omega} = 2\pi\sqrt{\frac{\ell}{g}}.
\]


\paragraph{Condizioni iniziali diverse.}
Consideriamo ora il caso in cui il pendolo venga lasciato partire dalla posizione di equilibrio, $\theta(0) = 0$, con velocità angolare iniziale nota. Partiamo dalla soluzione generale:
\[
\theta(t)=\theta_0\cos(\omega t+\varphi).
\]

Imponiamo le condizioni iniziali:
\[
\theta(0)=0,
\qquad
\left.\frac{d\theta}{dt}\right|_{t=0}=\Omega.
\]

\[
\theta(0)=\theta_0\cos\varphi=0
\quad\Longrightarrow\quad
\cos\varphi=0
\quad\Longrightarrow\quad
\varphi=\frac{\pi}{2}.
\]

Deriviamo:
\[
\frac{d\theta}{dt}=-\omega\theta_0\sin(\omega t+\varphi).
\]
Valutando\footnote{In questo contesto, valutare significa sostituire il valore di $t$ nell'espressione della derivata.} in $t=0$:
\[
\left.\frac{d\theta}{dt}\right|_{t=0}
=-\omega\theta_0\sin\varphi=\Omega.
\]
Con $\varphi=\frac{\pi}{2}$ si ha $\sin\varphi=1$, quindi:
\[
-\omega\theta_0=\Omega
\quad\Longrightarrow\quad
\theta_0=-\frac{\Omega}{\omega}.
\]

Allora:
\[
\theta(t)=\theta_0\cos\!\left(\omega t+\frac{\pi}{2}\right)
=-\frac{\Omega}{\omega}\cos\!\left(\omega t+\frac{\pi}{2}\right)
=\frac{\Omega}{\omega}\sin(\omega t).
\]

\section{Lavoro}

Il \textbf{lavoro} di una forza misura l'effetto della forza quando il punto materiale subisce uno spostamento. Nel caso in cui la forza $\vec{F}$ sia \textbf{costante} e lo spostamento complessivo sia $\Delta \vec{s}$, il lavoro si definisce come prodotto scalare:
\begin{equation}
\boxed{
L = \vec{F}\cdot \Delta \vec{s} = F\,\Delta s \cos\alpha,
}
\end{equation}
\addcontentsline{equ}{myequations}{Lavoro di una forza costante}
dove $\alpha$ è l'angolo tra la direzione della forza e quella dello spostamento.

Da questa definizione seguono alcuni casi importanti:
\begin{itemize}
    \item $L$ è \textbf{massimo} se $\vec{F}$ è parallela a $\Delta\vec{s}$ ($\alpha=0$);
    \item $L=0$ se $\vec{F}$ è perpendicolare a $\Delta\vec{s}$ ($\alpha=\frac{\pi}{2}$);
    \item $L$ è \textbf{minimo} (negativo) se $\vec{F}$ e $\Delta\vec{s}$ sono antiparalleli ($\alpha=\pi$).
\end{itemize}

\subsection{Esempi qualitativi: peso e attrito}

La forza peso può compiere lavoro positivo o negativo a seconda del verso dello spostamento (verso il basso o verso l'alto). La forza di attrito, essendo diretta in verso opposto al moto, compie invece \textbf{lavoro negativo}:
\[
L_{\text{attr}}<0.
\]

\subsection{Generalizzazione: forze non costanti}

Se la forza non è costante, la definizione si estende considerando uno spostamento infinitesimo $d\vec{s}$. Si definisce il \textbf{lavoro elementare}:
\begin{equation}
\boxed{
dL = \vec{F}\cdot d\vec{s}.
}
\end{equation}
\addcontentsline{equ}{myequations}{Lavoro elementare di una forza}
Il \textbf{lavoro totale} compiuto dalla forza nello spostamento da un punto $A$ a un punto $B$ è:
\begin{equation}
\boxed{
L_{A\to B}=\int_{A}^{B} \vec{F}\cdot d\vec{s}.
}
\end{equation}
\addcontentsline{equ}{myequations}{Lavoro totale di una forza}
Questa quantità rappresenta il lavoro complessivo compiuto dalla forza per portare il punto materiale da $A$ a $B$ lungo una certa traiettoria.

\subsection{Lavoro della forza peso: lancio verso l'alto e caduta verso il basso}
Consideriamo un moto verticale e scegliamo l'asse $y$ verso l'alto. La forza peso vale:
\[
\vec{P}=(0,-mg,0).
\]

\paragraph{Lancio verso l'alto.}
Durante lo spostamento verso l'alto, $d\vec{s}$ è diretto verso l'alto mentre $\vec{P}$ è verso il basso: sono antiparalleli e dunque il lavoro è negativo.
Scrivendo in forma scalare lungo $y$:
\[
dL=\vec{P}\cdot d\vec{s}=P_y\,dy=(-mg)\,dy.
\]
Integrando da $y=0$ a $y=h$:
\[
L_{0\to h}=\int_{0}^{h}(-mg)\,dy=-mgh<0.
\]

\paragraph{Caduta verso il basso.}
Durante la caduta, lo spostamento è verso il basso e quindi è parallelo alla forza peso: il lavoro è positivo.
Se il corpo scende di un dislivello $h$:
\[
L_{h\to 0}=+mgh>0.
\]

\medskip
In sintesi, per la forza peso vale:
\begin{equation}
\boxed{
L_{\text{peso}} = m g (y_A - y_B),
}
\end{equation}
\addcontentsline{equ}{myequations}{Lavoro della forza peso}
dove $y_A$ e $y_B$ sono le quote iniziale e finale. Questo significa che il lavoro della forza peso dipende solo dalla variazione di quota,

\subsection{Ricavare il lavoro usando la legge oraria}

Nel moto verticale, $d\vec{s}$ è lungo $y$ e vale $dy = v_y(t)\,dt$. Pertanto:
\[
dL = P_y\,dy = (-mg)\,v_y(t)\,dt.
\]
Nel caso di lancio verso l'alto, con $v_y(t)=v_0-g t$, si integra fino all'istante $t_f$ in cui $v_y(t_f)=0$:
\[
L=\int_{0}^{t_f}(-mg)\,v_y(t)\,dt = -mg\int_{0}^{t_f}(v_0 - g t)\,dt = -mg\left[v_0 t - \frac{1}{2} g t^2\right]_{0}^{t_f}.
\]
Il risultato coincide con $L=-mgh$, dove $h$ è la quota massima raggiunta. Questo mostra che il lavoro della forza peso può essere espresso in funzione del dislivello.

\subsection{Lavoro della forza peso nel pendolo}

\begin{figure}[htbp]
    \centering
    \includegraphics[width=0.8\textwidth]{images/lavoro_pendolo.png}
    \caption{Pendolo: la tensione non compie lavoro lungo la traiettoria; il lavoro è dovuto alla componente tangenziale del peso.}
    \label{fig:lavoro_pendolo}
\end{figure}

Nel pendolo semplice il punto materiale si muove lungo un arco di circonferenza. La tensione del filo è radiale e dunque è perpendicolare allo spostamento tangenziale: \textbf{non compie lavoro}. Il lavoro lungo la traiettoria è quindi dovuto alla sola componente tangenziale del peso.

Indicando con $\theta$ l'angolo rispetto alla verticale e con $s$ l'arco, vale:
\[
ds = \ell\,d\theta.
\]
La componente tangenziale della forza peso (opposta all'aumento di $\theta$) ha modulo $mg\sin\theta$, quindi:
\[
dL = \vec{P}\cdot d\vec{s} = (mg\sin\theta)\,ds = mg\sin\theta\,\ell\,d\theta.
\]
Integrando, ad esempio, dalla posizione iniziale $\theta=\theta_0$ alla posizione finale $\theta=0$:
\[
L_{\theta_0\to 0}=mg\ell\int_{\theta_0}^{0}\sin\theta\,d\theta
=mg\ell\bigl(\cos 0-\cos\theta_0\bigr)
=mg\ell\,(1-\cos\theta_0)>0.
\]

Poiché la variazione di quota tra le due posizioni è $h=\ell(1-\cos\theta_0)$, si ottiene ancora:
\begin{equation}
\boxed{
L_{\text{peso}} = mgh.
}
\end{equation}
\addcontentsline{equ}{myequations}{Lavoro della forza peso nel pendolo}
Questo evidenzia un fatto importante: il lavoro della forza peso \textbf{non dipende dalla traiettoria}, ma solo dai punti iniziale e finale (cioè dal dislivello).

\section{Teorema del lavoro e dell'energia cinetica}

\subsection{Energia cinetica}

Quando un punto materiale di massa $m$ si muove con velocità di modulo $v$, gli associamo una grandezza che misura ``quanto moto'' possiede: l'\textbf{energia cinetica}. Essa è definita come
\begin{equation}
\boxed{
K = \frac{1}{2} m v^2.
}
\end{equation}
\addcontentsline{equ}{myequations}{Energia cinetica}
L'energia cinetica è sempre positiva (o nulla se il corpo è fermo). Se il corpo aumenta la propria velocità, allora $K$ aumenta; se invece rallenta, $K$ diminuisce.

\subsection{Enunciato del teorema}

Il \textbf{teorema del lavoro e dell'energia cinetica} afferma che:

\begin{quote}
Il lavoro totale compiuto dalla forza risultante su un punto materiale nello spostamento da una posizione iniziale $A$ a una posizione finale $B$ è uguale alla variazione della sua energia cinetica.
\end{quote}

In formule:
\begin{equation}
\boxed{
L_{A\to B}=K_B-K_A.
}
\end{equation}
\addcontentsline{equ}{myequations}{Teorema del lavoro e dell'energia cinetica}

Di conseguenza:
\begin{itemize}
\item se $L_{A\to B}>0$, allora $K$ aumenta (il corpo accelera);
\item se $L_{A\to B}<0$, allora $K$ diminuisce (il corpo rallenta);
\item se $L_{A\to B}=0$, allora $K$ resta costante.
\end{itemize}

\paragraph{Dimostrazione.}

Consideriamo un lavoro elementare $dL$ compiuto dalla forza risultante $\vec{F}$ durante uno spostamento infinitesimo $d\vec{s}$:
\[
dL=\vec{F}\cdot d\vec{s}.
\]

Poiché la velocità è
\[
\vec{v}=\frac{d\vec{s}}{dt}
\qquad \Longrightarrow \qquad
d\vec{s}=\vec{v}\,dt,
\]
si ottiene:
\[
dL=\vec{F}\cdot \vec{v}\,dt.
\]

Usando il secondo principio della dinamica $\vec{F}=m\dfrac{d\vec{v}}{dt}$:
\[
dL=m\frac{d\vec{v}}{dt}\cdot \vec{v}\,dt
= m\,\vec{v}\cdot d\vec{v}.
\]

Ora osserviamo che, dato $d(f(x)) = f'(x)\,dx$ (derivata di una funzione composta), si ha:
\[
d(v^2)=d(\vec{v}\cdot\vec{v})
=2\,\vec{v}\cdot d\vec{v}
\qquad \Longrightarrow \qquad
\vec{v}\cdot d\vec{v}=\frac{1}{2}\,d(v^2).
\]
Quindi:
\[
dL = m\left(\frac{1}{2}\,d(v^2)\right)
= d\!\left(\frac{1}{2}mv^2\right)
= dK.
\]

Integrando tra $A$ e $B$:
\[
L_{A\to B}=\int_A^B dL
=\int_A^B dK
=K_B-K_A,
\]
che è esattamente il teorema del lavoro e dell'energia cinetica.

\paragraph{Osservazioni.}

Il teorema del lavoro e dell'energia cinetica mette in evidenza un fatto molto intuitivo: il lavoro della forza risultante misura quanto cambia la velocità del corpo. Infatti, se il lavoro totale nello spostamento $A\to B$ è positivo ($L_{A\to B}>0$), allora l'energia cinetica aumenta e quindi aumenta anche il modulo della velocità; viceversa, se il lavoro è negativo ($L_{A\to B}<0$), l'energia cinetica diminuisce e il corpo rallenta. Nel caso limite $L_{A\to B}=0$ l'energia cinetica resta costante: ciò significa che il modulo della velocità non cambia.

Un aspetto importante è che il teorema vale per \emph{qualunque tipo di forza}: non è necessario che la forza sia costante, né che il moto sia rettilineo. L'unica cosa che cambia da un problema all'altro è il modo in cui si calcola il lavoro $L_{A\to B}$, cioè l'integrale del prodotto scalare $\vec{F}\cdot d\vec{s}$ lungo la traiettoria. Per questo motivo, il teorema è uno strumento molto potente: permette di collegare direttamente le forze al cambiamento di velocità, spesso evitando di risolvere esplicitamente le equazioni del moto.

\paragraph{Esempio 1: lavoro della forza peso.}
Considerando il lavoro della forza peso calcolato in precedenza:
\[
L_{\text{peso}} = m g (y_A - y_B),
\]
il teorema del lavoro e dell'energia cinetica si traduce in:
\[
m g (y_A - y_B) = K_B - K_A.
\]
Questo risultato mostra che, quando un corpo si sposta verticalmente sotto l'azione della forza peso, la variazione della sua energia cinetica dipende solo dalla differenza di quota tra i punti iniziale e finale.

\section{Legge di conservazione dell'energia meccanica}

\subsection{Forze conservative}

Una forza si dice \textbf{conservativa} se il lavoro da essa compiuto nello spostamento di un punto materiale dipende solo dalla posizione iniziale e finale e non dalla traiettoria seguita.

In questo caso, il lavoro della forza nello spostamento da $A$ a $B$ può essere scritto come:
\[
L_{A\to B} = U_A - U_B,
\]
dove $U$ è detta \textbf{energia potenziale} associata alla forza. L'energia potenziale è una misura del lavoro che la forza può compiere in un certo punto dello spazio. Questa relazione \emph{può essere assunta come definizione di forza conservativa}.

\subsection{Energia meccanica}

Dal teorema del lavoro e dell'energia cinetica si ha:
\[
K_B - K_A = L_{A\to B}.
\]
Se la forza è conservativa, allora:
\[
K_B - K_A = U_A - U_B.
\]
Riordinando i termini:
\[
K_B + U_B = K_A + U_A.
\]

Definendo l'\textbf{energia meccanica} come:
\begin{equation}
\boxed{
E = K + U,
}
\end{equation}
\addcontentsline{equ}{myequations}{Energia meccanica}

si ottiene la \textbf{legge di conservazione dell'energia meccanica}:
\[
E_B = E_A,
\]
ovvero l'energia meccanica di un sistema soggetto a sole forze conservative rimane costante durante il moto.

\subsection{Energia potenziale della forza peso}

In prossimità della superficie terrestre, la forza peso può essere considerata costante in modulo e direzione ed è una forza conservativa. Scegliendo un asse verticale $y$ orientato verso l’alto, la forza peso agisce lungo la direzione negativa dell’asse e vale
\[
\vec{F}_{\text{peso}} = -mg\,\hat{y}.
\]

Il lavoro compiuto dalla forza peso nello spostamento verticale di un corpo dalla quota $y_A$ alla quota $y_B$ è dato da
\[
L_{\text{peso}} = \int_{y_A}^{y_B} \vec{F}_{\text{peso}} \cdot d\vec{r}
= \int_{y_A}^{y_B} (-mg)\,dy
= mg(y_A - y_B).
\]

Poiché la forza peso è una forza conservativa, il lavoro che essa compie nello spostamento di un corpo tra due posizioni dipende esclusivamente dalle posizioni iniziale e finale. È quindi possibile introdurre una funzione scalare $U(y)$, detta energia potenziale, tale che il lavoro della forza peso nello spostamento dalla quota $y_A$ alla quota $y_B$ sia uguale all’opposto della variazione di energia potenziale:
\[
L_{\text{peso}} = -\Delta U = U(y_A) - U(y_B).
\]

D’altra parte, calcolando esplicitamente il lavoro della forza peso in un campo gravitazionale uniforme, si ottiene
\[
L_{\text{peso}} = mg\,(y_A - y_B).
\]

Confrontando le due espressioni del lavoro, segue che la funzione energia potenziale deve soddisfare la relazione
\[
U(y_A) - U(y_B) = mg\,(y_A - y_B),
\]
valida per qualunque scelta delle quote $y_A$ e $y_B$. Ciò implica che la variazione di $U$ è proporzionale alla variazione della coordinata verticale $y$, e quindi che $U(y)$ deve essere una funzione lineare di $y$. Ne segue che l’energia potenziale associata alla forza peso può essere scritta nella forma
\[
U(y) = mgy + C,
\]
dove $C$ è una costante additiva arbitraria.

La presenza della costante riflette il fatto che il valore assoluto dell’energia potenziale non è fisicamente osservabile: solo le differenze di energia potenziale hanno significato fisico. La scelta di $C$ equivale pertanto a fissare il livello di riferimento per l’energia potenziale. Adottando convenzionalmente come riferimento la quota $y=0$ e imponendo la condizione $U(0)=0$, si ottiene infine
\[
U(y) = mgy.
\]

\subsection{Energia potenziale della forza elastica}

Anche la forza elastica è una forza conservativa. In una dimensione, la forza elastica è data da:
\[
F_e = -kx.
\]

Il lavoro della forza elastica nello spostamento dalla posizione $0$ alla posizione $x$ è:
\[
L_{0\to x} = \int_0^x \vec{F}_e \cdot d\vec{x}
= \int_0^x (-kx)\,dx
= -\frac{1}{2}kx^2.
\]

Poiché:
\[
L_{0\to x} = U(0) - U(x),
\]
ponendo $U(0)=0$ si ottiene:
\[
U(x) = \frac{1}{2}kx^2.
\]

\subsection{Applicazioni della conservazione dell'energia meccanica}

\paragraph{Lancio verticale verso l'alto.}

Consideriamo un punto materiale lanciato verticalmente verso l'alto con velocità iniziale $v_0$. Applicando la conservazione dell'energia meccanica tra l'istante iniziale e il punto di massima quota:
\[
E_i = E_f
\quad\Longrightarrow\quad
K_i + U_i = K_f + U_f.
\]
Poiché nel punto più alto la velocità è nulla, possiamo riscriverre la relazione come:
\[
\frac{1}{2}mv_0^2 = mgh,
\]
da cui:
\[
h = \frac{v_0^2}{2g}.
\]

\paragraph{Caduta libera.}

Consideriamo un punto materiale che cade da un'altezza $h$. All'istante iniziale:
\[
K_i = 0, \qquad U_i = mgh,
\]
mentre al suolo:
\[
U_f = 0, \qquad K_f = \frac{1}{2}mv_f^2.
\]
Dalla conservazione dell'energia:
\[
mgh = \frac{1}{2}mv_f^2
\qquad \Longrightarrow \qquad
v_f = \sqrt{2gh}.
\]

\paragraph{Moto parabolico.}

Nel moto parabolico, se si trascurano gli attriti, agisce solo la forza peso, che è conservativa. Pertanto l'energia meccanica si conserva.

Se il punto materiale parte e arriva alla stessa quota:
\[
U_i = U_f
\quad\Longrightarrow\quad
K_i = K_f
\quad\Longrightarrow\quad
|v_i| = |v_f|.
\]

\paragraph{Pendolo semplice.}

Consideriamo un pendolo semplice di lunghezza $\ell$, lasciato partire da un angolo iniziale $\theta_0$. Applicando la conservazione dell'energia meccanica tra la posizione iniziale e la posizione più bassa della traiettoria:
\[
E_i = E_f.
\]

L'energia potenziale iniziale è:
\[
U_i = mgh,
\qquad
h = \ell(1-\cos\theta_0),
\]
mentre l'energia cinetica iniziale è nulla. Nella posizione più bassa:
\[
U_f = 0, \qquad K_f = \frac{1}{2}mv^2.
\]
Dalla conservazione dell'energia:
\[
m g \ell (1-\cos\theta_0) = \frac{1}{2} m v^2,
\]
da cui:
\begin{equation}
\boxed{
v = \sqrt{2g\ell(1-\cos\theta_0)}.
}
\end{equation}
\addcontentsline{equ}{myequations}{Velocità del pendolo con Energia meccanica}

Questo risultato è valido per qualunque valore dell'angolo iniziale $\theta_0$ e mostra che, durante il moto, l'energia cinetica viene continuamente convertita in energia potenziale e viceversa, in modo tale che la loro somma rimanga costante.

\section{Impulso della forza e quantità di moto}

\subsection{Quantità di moto}

Si definisce \textbf{quantità di moto} di un punto materiale la grandezza vettoriale:
\begin{equation}
\boxed{
\vec{p} = m \vec{v}.
}
\end{equation}
\addcontentsline{equ}{myequations}{Quantità di moto}

La quantità di moto è direttamente proporzionale alla massa del corpo e alla sua velocità e rappresenta una misura dello stato di moto del punto materiale.

\subsection{Impulso della forza}

Partendo dal secondo principio della dinamica, nella forma:
\[
\vec{F} = m \frac{d\vec{v}}{dt},
\]
si moltiplica entrambi i membri per $dt$:
\[
\vec{F}\,dt = m\, d\vec{v}.
\]

Integrando tra un istante iniziale $t_i$ e un istante finale $t_f$:
\[
\int_{t_i}^{t_f} \vec{F}\,dt
= m \int_{\vec{v}_i}^{\vec{v}_f} d\vec{v}
= m(\vec{v}_f - \vec{v}_i).
\]

Si definisce \textbf{impulso della forza} la grandezza:
\begin{equation}
\boxed{
\vec{I} = \int_{t_i}^{t_f} \vec{F}\,dt.
}
\end{equation}
\addcontentsline{equ}{myequations}{Impulso della forza}

Pertanto si ottiene:
\[
\vec{I} = m\vec{v}_f - m\vec{v}_i
= \vec{p}_f - \vec{p}_i
= \Delta \vec{p}.
\]

Questo risultato prende il nome di \textbf{teorema dell'impulso e della quantità di moto} (o teorema dell'impulso della forza) e afferma che:
\begin{quote}
L'impulso della forza risultante agente su un punto materiale è uguale alla variazione della sua quantità di moto.
\end{quote}

\subsection{Forma generale del secondo principio della dinamica}

Poiché la quantità di moto è definita come $\vec{p} = m\vec{v} \Rightarrow \vec{v} = \vec{p}/{m}$, il secondo principio della dinamica può essere riscritto nella forma:
\[
\vec{F} = m\vec{a} = m \frac{d\vec{v}}{dt} = m \frac{d}{dt}\left(\frac{\vec{p}}{m}\right) = \frac{d\vec{p}}{dt}.
\]

Questa espressione rappresenta la forma più generale del secondo principio della dinamica ed è valida anche nel caso in cui la massa del corpo non sia costante.

\subsection{Conservazione della quantità di moto}

Se la forza risultante agente su un punto materiale è nulla, si ha:
\[
\vec{F} = 0
\quad\Longrightarrow\quad
\frac{d\vec{p}}{dt} = 0
\quad\Longrightarrow\quad
\vec{p} = \text{costante}.
\]

In un sistema di più punti materiali, la quantità di moto totale è definita come:
\[
\vec{P} = \sum_i \vec{p}_i.
\]

Se il sistema è \textbf{isolato}, cioè se la risultante delle forze esterne è nulla ($\vec{F}_{\text{ext}}=0$), allora:
\[
\frac{d\vec{P}}{dt} = 0
\quad\Longrightarrow\quad
\vec{P} = \text{costante}.
\]

La quantità di moto totale di un sistema isolato si conserva. Questa legge di conservazione è una conseguenza diretta dell'invarianza delle leggi fisiche per traslazioni nello spazio.

\subsection{Urto di una biglia contro una parete}

\begin{figure}[htbp]
    \centering
    \includegraphics[width=0.6\textwidth]{images/urto_biglia_parete.png}
    \caption{Urto di una biglia contro una parete rigida: gli angoli di incidenza e riflessione sono uguali.}
    \label{fig:urto_biglia_parete}
\end{figure}

Consideriamo l'urto di una biglia contro una parete rigida. Indichiamo con $\theta_i$ l'angolo di incidenza e con $\theta_f$ l'angolo di riflessione, entrambi misurati rispetto alla normale alla parete.

Si assume che la parete abbia massa infinita e sia priva di attrito. Di conseguenza, la forza esercitata dalla parete sulla biglia è diretta esclusivamente lungo la direzione perpendicolare alla parete (direzione normale), mentre non sono presenti componenti parallele alla parete.

\paragraph{Quantità di moto prima e dopo l'urto.}

Indichiamo con $\vec{p}_i = m\vec{v}_i$ la quantità di moto della biglia prima dell'urto e con $\vec{p}_f = m\vec{v}_f$ la quantità di moto dopo l'urto.

Poiché non agiscono forze di attrito, la componente parallela alla parete della forza esercitata sulla biglia è nulla. In particolare, lungo la direzione $y$:
\[
F_y = 0
\quad \Longrightarrow \quad
\Delta p_y = 0
\quad \Longrightarrow \quad
p_{y,f} = p_{y,i}
\quad \Longrightarrow \quad
v_{y,f} = v_{y,i}.
\]

\paragraph{Urto elastico.}

Sperimentalmente si osserva che, nell'urto di una biglia contro una parete rigida, l'energia cinetica si conserva:
\[
K_i = K_f
\quad \Longrightarrow \quad
\frac{1}{2}m v_i^2 = \frac{1}{2}m v_f^2
\quad \Longrightarrow \quad
v_i^2 = v_f^2.
\]

Scrivendo il modulo della velocità in termini delle componenti:
\[
v_i^2 = v_{x,i}^2 + v_{y,i}^2,
\qquad
v_f^2 = v_{x,f}^2 + v_{y,f}^2,
\]
e usando il fatto che $v_{y,f} = v_{y,i}$, si ottiene:
\[
v_{x,f}^2 = v_{x,i}^2
\quad \Longrightarrow \quad
v_{x,f} = -v_{x,i}.
\]

La componente della velocità perpendicolare alla parete cambia segno, mentre quella parallela resta invariata.

\paragraph{Angoli di incidenza e riflessione.}

\begin{figure}[htbp]
    \centering
    \includegraphics[width=0.6\textwidth]{images/angoli_incidenza_riflessione.png}
    \caption{Schema dell'urto elastico di una biglia contro una parete liscia: sono indicati l'angolo di incidenza $\theta_i$, l'angolo di riflessione $\theta_f$ (uguale a $\theta_i$), l'asse $x$ perpendicolare alla parete e le direzioni delle velocità prima e dopo l'urto.}
    \label{fig:angoli_incidenza_riflessione}
\end{figure}

Dalla definizione di angolo di incidenza e riflessione si ha:
\[
\cos\theta_i = \frac{v_{x,i}}{v_i},
\qquad
\cos\theta_f = \frac{v_{x,f}}{v_f}.
\]

Poiché $v_f = v_i$ e $v_{x,f} = -v_{x,i}$:
\[
\cos\theta_f = -\cos\theta_i = \cos(\pi - \theta_i).
\]
Ne segue che:
\[
\theta_f = \theta_i.
\]

Questo risultato mostra che, nell'urto elastico contro una parete liscia, l'angolo di riflessione è uguale all'angolo di incidenza.

\paragraph{Impulso della forza esercitata dalla parete.}

L'urto avviene in un intervallo di tempo molto breve $\tau$. L'impulso della forza esercitata dalla parete sulla biglia è:
\[
\vec{I} = \vec{p}_f - \vec{p}_i.
\]

Lungo la direzione perpendicolare alla parete si ha:
\[
I_x = p_{x,f} - p_{x,i}
= m v_{x,f} - m v_{x,i}
= -2m v_{x,i}
= 2m v \cos\theta_i.
\]

Per definizione di impulso:
\[
I_x = \int_0^{\tau} F_x\,dt.
\]

Definendo la forza media lungo $x$ come:
\[
\langle F_x \rangle = \frac{1}{\tau} \int_0^{\tau} F_x\,dt,
\]
si ottiene:
\begin{equation}
\boxed{
\langle F_x \rangle = \frac{2m v \cos\theta_i}{\tau}.
}
\end{equation}
\addcontentsline{equ}{myequations}{Forza media esercitata dalla parete sulla biglia}

La forza media esercitata dalla parete sulla biglia è quindi diretta perpendicolarmente alla parete.

\paragraph{Urti impulsivi e conservazione della quantità di moto.}

L'urto di una biglia contro una parete è un \textbf{processo impulsivo}, cioè un processo in cui la quantità di moto del punto materiale viene modificata per effetto di una forza molto intensa agente per un tempo molto breve.

Durante un processo impulsivo è possibile trascurare l'effetto delle forze esterne (come la forza peso) e considerare il sistema isolato. In tal caso:
\[
\vec{F}_{\text{ext}} \simeq 0
\quad \Longrightarrow \quad
\vec{P}_{\text{tot}} = \text{costante}.
\]

\paragraph{Urti elastici e anelastici.}

Gli urti si classificano in:
\begin{itemize}
\item \textbf{urti elastici}, nei quali si conserva l'energia cinetica totale del sistema;
\item \textbf{urti anelastici}, nei quali l'energia cinetica totale non si conserva.
\end{itemize}

\subsection{Urto elastico tra due particelle}

\begin{figure}[htbp]
    \centering
    \includegraphics[width=0.8\textwidth]{images/urto_elastico_due_particelle.png}
    \caption{Urto elastico unidimensionale tra due particelle: la particella di massa $m_1$ si muove inizialmente lungo l’asse $x$ con velocità $\vec{v}_1$, mentre la particella di massa $m_2$ è inizialmente ferma ($\vec{v}_2=0$).}
    \label{fig:urto_elastico_due_particelle}
\end{figure}

Consideriamo un urto elastico unidimensionale tra due particelle di masse $m_1$ e $m_2$. La particella 1 si muove inizialmente lungo l’asse $x$ con velocità $v_1$, mentre la particella 2 è inizialmente ferma:
\[
v_2 = 0.
\]
Dopo l’urto, le velocità delle due particelle diventano rispettivamente $V_1$ e $V_2$. Vogliamo determinare tali velocità finali.

\paragraph{Equazioni di conservazione.}

Poiché il sistema è isolato, la quantità di moto totale si conserva:
\[
p_{\text{tot}} = \text{costante}
\quad \Longrightarrow \quad
m_1 v_1 + m_2 v_2 = m_1 V_1 + m_2 V_2.
\]
Dato che $v_2=0$, si ottiene:
\[
m_1 v_1 = m_1 V_1 + m_2 V_2
\quad \Longrightarrow \quad
v_1 = V_1 + \frac{m_2}{m_1} V_2.
\]

Essendo l’urto elastico, si conserva anche l’energia cinetica:
\[
K_i = K_f
\quad \Longrightarrow \quad
\frac{1}{2}m_1 v_1^2
= \frac{1}{2}m_1 V_1^2 + \frac{1}{2}m_2 V_2^2.
\]
Dividendo per $\tfrac{1}{2}m_1$:
\[
v_1^2 = V_1^2 + \frac{m_2}{m_1} V_2^2.
\]

\paragraph{Risoluzione del sistema.}

Sostituendo l’espressione di $V_1$ ricavata dalla conservazione della quantità di moto:
\[
V_1 = v_1 - \frac{m_2}{m_1} V_2,
\]
nell’equazione dell’energia, si ottiene:
\[
v_1^2 =
\left(v_1 - \frac{m_2}{m_1} V_2\right)^2
+ \frac{m_2}{m_1} V_2^2.
\]

Sviluppando:
\[
v_1^2 =
v_1^2 - 2v_1 \frac{m_2}{m_1} V_2
+ \frac{m_2^2}{m_1^2} V_2^2
+ \frac{m_2}{m_1} V_2^2.
\]

Semplificando e raccogliendo:
\[
\frac{m_2}{m_1} V_2
\left(
\frac{m_2}{m_1} V_2 - 2v_1 + V_2
\right) = 0.
\]

La soluzione $V_2=0$ corrisponde all’assenza dell’urto e viene scartata. Rimane quindi:
\[
\left(\frac{m_1}{m_2} + 1\right)V_2 = 2v_1
\quad \Longrightarrow \quad
V_2 = \frac{2m_1}{m_1+m_2} v_1.
\]

Sostituendo in $V_1$:
\[
V_1 = v_1 - \frac{m_2}{m_1} V_2 
= v_1 - \frac{m_2}{m_1} \cdot \frac{2m_1}{m_1+m_2} v_1 
= v_1 - \frac{2m_2}{m_1+m_2} v_1
= \frac{m_1 - m_2}{m_1 + m_2} v_1.
\]

\paragraph{Risultati finali.}

Le velocità finali delle due particelle dopo l’urto elastico sono quindi:
\begin{equation}
\boxed{
V_1 = \frac{m_1 - m_2}{m_1 + m_2} v_1,
\qquad
V_2 = \frac{2m_1}{m_1 + m_2} v_1.
}
\end{equation}
\addcontentsline{equ}{myequations}{Velocità finali dopo urto elastico tra due particelle}

\paragraph{Casi limite.}

\begin{itemize}
\item \textbf{$m_1 = m_2$:}  
\[
V_1 = 0,
\qquad
V_2 = v_1.
\]
Le due particelle si scambiano la velocità.

\item \textbf{$m_2 \gg m_1$:}  
\[
V_2 \simeq 0,
\qquad
V_1 \simeq -v_1.
\]
La particella 2 rimane praticamente ferma, mentre la particella 1 viene riflessa.

\item \textbf{$m_1 \gg m_2$:}  
\[
V_1 \simeq v_1,
\qquad
V_2 \simeq 2v_1.
\]
La particella 1 trasferisce parte del moto alla particella 2, che viene ``lanciata in avanti''.
\end{itemize}

\subsection{Urto elastico con entrambe le particelle in movimento}

\begin{figure}[htbp]
    \centering
    \includegraphics[width=0.8\textwidth]{images/urto_elastico_due_particelle_generale.png}
    \caption{Urto elastico tra due particelle osservato in due sistemi di riferimento: a sinistra il sistema dell’osservatore, a destra il sistema solidale con la particella di massa $m_2$.}
    \label{fig:urto_elastico_due_particelle_generale}
\end{figure}

Nel caso generale in cui entrambe le particelle siano inizialmente in movimento con velocità $v_1$ e $v_2$, è conveniente considerare un sistema di riferimento solidale con la particella 2. In tale sistema, la velocità iniziale della particella 2 è nulla.

Applicando le formule precedenti e tornando poi al sistema di riferimento dell’osservatore, si ottengono le velocità finali:
\begin{equation}
\boxed{
V_1 = \frac{m_1 - m_2}{m_1 + m_2} v_1
+ \frac{2m_2}{m_1 + m_2} v_2, 
\qquad
V_2 = \frac{2m_1}{m_1 + m_2} v_1
+ \frac{m_2 - m_1}{m_1 + m_2} v_2.
}
\end{equation}
\addcontentsline{equ}{myequations}{Velocità finali dopo urto elastico tra due particelle in movimento}

Queste espressioni descrivono completamente l’urto elastico unidimensionale tra due particelle.

\subsection{Urto perfettamente anelastico}

Si ha un \textbf{urto perfettamente anelastico} quando due punti materiali, dopo una collisione, proseguono il moto come se fossero un unico punto materiale. In altre parole, dopo l’urto le due particelle hanno la stessa velocità finale.

In un urto perfettamente anelastico non si conserva l’energia cinetica totale del sistema, mentre si conserva la quantità di moto totale.

\paragraph{Conservazione della quantità di moto.}

Consideriamo due particelle di masse $m_1$ e $m_2$ che si muovono lungo una stessa direzione. Supponiamo che la particella 2 sia inizialmente ferma:
\[
v_2 = 0.
\]

Poiché il sistema è isolato, la quantità di moto totale si conserva:
\[
\vec{p}_{\text{tot}} = \text{costante}.
\]

Indicando con $V$ la velocità comune delle due particelle dopo l’urto, si ha:
\[
m_1 v_1 + m_2 v_2 = m_1 V + m_2 V.
\]

Dato che $v_2 = 0$:
\begin{equation}
m_1 v_1 = (m_1 + m_2) V
\quad \Longrightarrow \quad
\boxed{
V = \frac{m_1}{m_1 + m_2} v_1.
}
\end{equation}
\addcontentsline{equ}{myequations}{Velocità finale dopo urto
perfettamente anelastico}
\paragraph{Osservazioni.}

\begin{itemize}
\item Se $m_2 \gg m_1$, allora $V \simeq 0$: la particella 1 si arresta quasi completamente dopo l’urto.
\item Se $m_1 \gg m_2$, allora $V \simeq v_1$: la velocità della particella 1 rimane praticamente invariata.
\end{itemize}

Questo tipo di urto rappresenta il caso di massima perdita di energia cinetica compatibile con la conservazione della quantità di moto.

\section{Legge della gravitazione universale di Newton}

La \textbf{legge della gravitazione universale} descrive l’interazione attrattiva tra due masse puntiformi. Essa afferma che due corpi di masse $m_1$ e $m_2$ si attraggono con una forza diretta lungo la congiungente\footnote{La congiungente è la linea immaginaria che unisce i centri di massa dei due corpi} dei due corpi e di modulo inversamente proporzionale al quadrato della loro distanza.

Indicando con $\vec{r}_{12}$ il vettore posizione che va dal corpo 2 al corpo 1, con modulo $r_{12}=|\vec{r}_{12}|$ e versore $\hat{r}$, la forza esercitata da $m_2$ su $m_1$ è:
\begin{equation}
\boxed{
\vec{F}_{12} = -G \frac{m_1 m_2}{r_{12}^2}\,\hat{r}.
}
\end{equation}
\addcontentsline{equ}{myequations}{Forza di gravitazione universale}

\begin{figure}[htbp]
    \centering
    \includegraphics[width=0.8\textwidth]{images/legge_gravitazione_universale.png}
    \caption{Due masse puntiformi $m_1$ e $m_2$ separate da una distanza $r_{12}$ si attraggono con una forza di gravitazione universale $\vec{F}_{12}$.}
    \label{fig:legge_gravitazione_universale}
\end{figure}

La forza di gravitazione universale presenta le seguenti caratteristiche:
\begin{itemize}
\item è sempre \textbf{attrattiva};
\item decresce con il \textbf{quadrato della distanza};
\item è una \textbf{forza conservativa};
\item la formula vale anche per corpi sferici omogenei, purché si consideri la regione esterna ai corpi.
\end{itemize}

\subsection{Dimostrazione della conservatività della forza di gravitazione universale}

Vogliamo dimostrare che la forza di gravitazione universale è una forza conservativa, cioè che il lavoro da essa compiuto dipende solo dalla posizione iniziale e finale e non dal percorso seguito.

\paragraph{Ipotesi.}

\begin{itemize}
\item Siano $m_1$ e $m_2$ le masse di due corpi che si attraggono gravitazionalmente.
\item Siano $\vec{r}_1$ e $\vec{r}_2$ i vettori posizione delle masse $m_1$ e $m_2$.
\item Sia $\vec{r}_{12} = \vec{r}_1 - \vec{r}_2$ il vettore posizione relativa, con modulo $r_{12} = |\vec{r}_{12}|$ e versore $\hat{r}$.
\end{itemize}

\paragraph{Tesi.}

Vogliamo dimostrare che la forza di gravitazione universale
\[
\vec{F}_{12} = -G \frac{m_1 m_2}{r_{12}^2}\,\hat{r}
\]
è una forza conservativa.

\paragraph{Dimostrazione.}

Dalla legge di gravitazione universale osserviamo che la forza $\vec{F}_{12}$ è sempre diretta lungo il versore $\hat{r}$, cioè lungo la congiungente dei due corpi. Essa è quindi una forza centrale e non possiede componenti perpendicolari allo spostamento radiale.

Per verificare la conservatività della forza, calcoliamo il lavoro elementare compiuto durante uno spostamento infinitesimo $d\vec{r}$. Per definizione:
\[
dL = \vec{F} \cdot d\vec{r}.
\]

Nel nostro caso:
\[
dL = \vec{F}_{12} \cdot d\vec{r}.
\]

Il vettore posizione può essere scritto come:
\[
\vec{r} = r\,\hat{r}.
\]
Derivando:
\[
d\vec{r} = d(r\hat{r}) = \hat{r}\,dr + r\,d\hat{r}.
\]

Sostituendo:
\[
dL = -G \frac{m_1 m_2}{r^2}\,\hat{r} \cdot (\hat{r}\,dr + r\,d\hat{r})
= -G \frac{m_1 m_2}{r^2}\left( dr + r\,\hat{r}\cdot d\hat{r} \right).
\]

Poiché:
\[
\hat{r}\cdot\hat{r} = 1
\quad \Longrightarrow \quad
d(\hat{r}\cdot\hat{r}) = 2\hat{r}\cdot d\hat{r} = 0
\quad \Longrightarrow \quad
\hat{r}\cdot d\hat{r} = 0,
\]
segue che:
\[
dL = -G \frac{m_1 m_2}{r^2}\,dr.
\]

Integrando tra una posizione iniziale $r_A$ e una posizione finale $r_B$:
\[
L_{A\to B}
= \int_{r_A}^{r_B} -G \frac{m_1 m_2}{r^2}\,dr
= -G m_1 m_2 \int_{r_A}^{r_B} \frac{1}{r^2}\,dr.
\]

Poiché:
\[
\int \frac{1}{r^2}\,dr = -\frac{1}{r},
\]
si ottiene:
\[
L_{A\to B}
= -G m_1 m_2 \left(-\frac{1}{r}\right)\Big|_{r_A}^{r_B}.
\]

Il lavoro dipende esclusivamente dalle posizioni iniziale e finale e non dal percorso seguito. Questo dimostra che la forza di gravitazione universale è una forza conservativa.

\subsection{Energia potenziale gravitazionale}

Poiché la forza di gravitazione universale è conservativa, è possibile associare ad essa un’energia potenziale $U(r)$. Per una forza centrale di tipo gravitazionale si ha:
\begin{equation}
\boxed{
U(r) = -G \frac{m_1 m_2}{r}.
}
\end{equation}
\addcontentsline{equ}{myequations}{Energia potenziale gravitazionale}

L’energia potenziale gravitazionale è definita a meno di una costante additiva; convenzionalmente si sceglie:
\[
U(\infty) = 0.
\]

Il segno negativo indica che la forza è attrattiva e che l’energia potenziale diminuisce avvicinando i due corpi.

\subsection{Forza peso come caso particolare}

La forza peso può essere ricavata come caso particolare della legge di gravitazione universale. Consideriamo un corpo di massa $m$ posto sulla superficie terrestre. La forza gravitazionale esercitata dalla Terra (di massa $M_T$) sul corpo vale:
\[
F = G \frac{M_T m}{R_T^2},
\]
dove $R_T$ è il raggio terrestre.

Ponendo:
\[
g = G \frac{M_T}{R_T^2},
\]
si ottiene:
\[
F = mg,
\]
che coincide con l’espressione della forza peso.

\subsection{Velocità di fuga}

La \textbf{velocità di fuga} è la velocità minima che deve essere impressa a un punto materiale affinché esso possa allontanarsi da un corpo celeste fino a distanza infinita con velocità finale nulla.

Consideriamo un corpo di massa $m$ sulla superficie di un corpo celeste di massa $M$ e raggio $R$. L’energia meccanica iniziale è:
\[
E_i = \frac{1}{2} m v_0^2 - G \frac{M m}{R}.
\]

All’infinito si ha:
\[
E_f = 0,
\]
poiché $v_f=0$ e $U(\infty)=0$.

Imponendo la conservazione dell’energia meccanica:
\[
E_i = E_f,
\]
si ottiene:
\[
\frac{1}{2} m v_0^2 - G \frac{M m}{R} = 0.
\]

Da cui:
\begin{equation}
\boxed{
v_0 = \sqrt{\frac{2GM}{R}}.
}
\end{equation}
\addcontentsline{equ}{myequations}{Velocità di fuga}

Questa è l’espressione della velocità di fuga, che dipende solo dalla massa e dal raggio del corpo celeste e non dalla massa del corpo lanciato.

\section*{Riferimenti}
I riferimenti per questo capitolo sono:
\begin{itemize}
    \item Materiale visto a lezione.
    \item Capitolo 2, 3, 4 e 5 del libro \citetitle{gasparini2019fisica} \cite{gasparini2019fisica}
\end{itemize}